\documentclass[journal,12pt,twocolumn]{IEEEtran}

\usepackage{setspace}
\usepackage{gensymb}
\singlespacing
\usepackage[cmex10]{amsmath}

\usepackage{amsthm}

\usepackage{mathrsfs}
\usepackage{txfonts}
\usepackage{stfloats}
\usepackage{bm}
\usepackage{cite}
\usepackage{cases}
\usepackage{subfig}

\usepackage{longtable}
\usepackage{multirow}

\usepackage{enumitem}
\usepackage{mathtools}
\usepackage{steinmetz}
\usepackage{tikz}
\usepackage{circuitikz}
\usepackage{verbatim}
\usepackage{tfrupee}
\usepackage[breaklinks=true]{hyperref}
\usepackage{graphicx}
\usepackage{tkz-euclide}
\usepackage{pgfplots}

\usetikzlibrary{calc,math}
\usepackage{listings}
    \usepackage{color}                                            %%
    \usepackage{array}                                            %%
    \usepackage{longtable}                                        %%
    \usepackage{calc}                                             %%
    \usepackage{multirow}                                         %%
    \usepackage{hhline}                                           %%
    \usepackage{ifthen}                                           %%
    \usepackage{lscape}     
\usepackage{multicol}
\usepackage{chngcntr}

\DeclareMathOperator*{\Res}{Res}

\renewcommand\thesection{\arabic{section}}
\renewcommand\thesubsection{\thesection.\arabic{subsection}}
\renewcommand\thesubsubsection{\thesubsection.\arabic{subsubsection}}

\renewcommand\thesectiondis{\arabic{section}}
\renewcommand\thesubsectiondis{\thesectiondis.\arabic{subsection}}
\renewcommand\thesubsubsectiondis{\thesubsectiondis.\arabic{subsubsection}}


\hyphenation{op-tical net-works semi-conduc-tor}
\def\inputGnumericTable{}                                 %%

\lstset{
%language=C,
frame=single, 
breaklines=true,
columns=fullflexible
}
\begin{document}


\newtheorem{theorem}{Theorem}[section]
\newtheorem{problem}{Problem}
\newtheorem{proposition}{Proposition}[section]
\newtheorem{lemma}{Lemma}[section]
\newtheorem{corollary}[theorem]{Corollary}
\newtheorem{example}{Example}[section]
\newtheorem{definition}[problem]{Definition}

\newcommand{\BEQA}{\begin{eqnarray}}
\newcommand{\EEQA}{\end{eqnarray}}
\newcommand{\define}{\stackrel{\triangle}{=}}
\bibliographystyle{IEEEtran}
\raggedbottom
\setlength{\parindent}{0pt}
\providecommand{\mbf}{\mathbf}
\providecommand{\pr}[1]{\ensuremath{\Pr\left(#1\right)}}
\providecommand{\qfunc}[1]{\ensuremath{Q\left(#1\right)}}
\providecommand{\sbrak}[1]{\ensuremath{{}\left[#1\right]}}
\providecommand{\lsbrak}[1]{\ensuremath{{}\left[#1\right.}}
\providecommand{\rsbrak}[1]{\ensuremath{{}\left.#1\right]}}
\providecommand{\brak}[1]{\ensuremath{\left(#1\right)}}
\providecommand{\lbrak}[1]{\ensuremath{\left(#1\right.}}
\providecommand{\rbrak}[1]{\ensuremath{\left.#1\right)}}
\providecommand{\cbrak}[1]{\ensuremath{\left\{#1\right\}}}
\providecommand{\lcbrak}[1]{\ensuremath{\left\{#1\right.}}
\providecommand{\rcbrak}[1]{\ensuremath{\left.#1\right\}}}
\theoremstyle{remark}
\newtheorem{rem}{Remark}
\newcommand{\sgn}{\mathop{\mathrm{sgn}}}
\providecommand{\abs}[1]{$\left\vert#1\right\vert$}
\providecommand{\res}[1]{\Res\displaylimits_{#1}} 
\providecommand{\norm}[1]{$\left\lVert#1\right\rVert$}
%\providecommand{\norm}[1]{\lVert#1\rVert}
\providecommand{\mtx}[1]{\mathbf{#1}}
\providecommand{\mean}[1]{E$\left[ #1 \right]$}
\providecommand{\fourier}{\overset{\mathcal{F}}{ \rightleftharpoons}}
%\providecommand{\hilbert}{\overset{\mathcal{H}}{ \rightleftharpoons}}
\providecommand{\system}{\overset{\mathcal{H}}{ \longleftrightarrow}}
	%\newcommand{\solution}[2]{\textbf{Solution:}{#1}}
\newcommand{\solution}{\noindent \textbf{Solution: }}
\newcommand{\cosec}{\,\text{cosec}\,}
\providecommand{\dec}[2]{\ensuremath{\overset{#1}{\underset{#2}{\gtrless}}}}
\newcommand{\myvec}[1]{\ensuremath{\begin{pmatrix}#1\end{pmatrix}}}
\newcommand{\mydet}[1]{\ensuremath{\begin{vmatrix}#1\end{vmatrix}}}
\numberwithin{equation}{subsection}
\makeatletter
\@addtoreset{figure}{problem}
\makeatother
\let\StandardTheFigure\thefigure
\let\vec\mathbf
\renewcommand{\thefigure}{\theproblem}
\def\putbox#1#2#3{\makebox[0in][l]{\makebox[#1][l]{}\raisebox{\baselineskip}[0in][0in]{\raisebox{#2}[0in][0in]{#3}}}}
     \def\rightbox#1{\makebox[0in][r]{#1}}
     \def\centbox#1{\makebox[0in]{#1}}
     \def\topbox#1{\raisebox{-\baselineskip}[0in][0in]{#1}}
     \def\midbox#1{\raisebox{-0.5\baselineskip}[0in][0in]{#1}}
\vspace{3cm}
\title{Assignment-2}
\author{Sushma - CS20BTECH11051}
\maketitle
\newpage
\bigskip
\renewcommand{\thefigure}{\theenumi}
\renewcommand{\thetable}{\theenumi}
Download all python codes from 
\begin{lstlisting}
https://github.com/Sushma-AI1103/AI1103-Assingment-2/blob/main/assingment_2.py
\end{lstlisting}

 \section{Problem}{79}-Suppose the random variable U has uniform distribution on [0,1] and X = -2$\ln(U)$ .Find the  density of X  . \\ \\
 
\section{Solutions:}
$U$ - uniformly distributed random variable on $\in$ [0,1]. \\ \\  \\ \\ 
Probability density function of $U$ is: \\ \\
\begin{tikzpicture}
\begin{axis}[
    axis lines = left,
    xlabel = $U$,
    ylabel = {$f_U(u)$},
]
%Below the pdf red uniformly distributed U  is defined
\addplot [
    domain=0:1, 
    samples=100, 
    color=red,
]
{1};
\end{axis}
%\draw[red] (6.8,0)--(6.8,2.8)
\end{tikzpicture}
 $X$ is given by :
\begin{align}
  X = -2 \ln(U) \\
\implies    0 \leq X \leq \infty
\end{align}
CDF of  $X$ is defined as \\
\begin{align}
    F_X(x) &= Pr(X \le x) \\
           &= Pr( -2 \ln(U)\le x) \\
           &= Pr(\ln(U) \ge( -x) /2)\\
           &= Pr (U \ge \exp(-x/2))\\
           &= 1 - Pr(U \le exp(-x/2)\\
           &= 1 - exp(-x/2) 
\end{align}

PDF of $X$ : \\
\begin{align}
    f_X(x) & = \frac{d (F_X (x) )}{dx} \\
           & = \frac{1}{2}  exp((-x)/2)
\end{align}

\begin{align}
\implies    f_X(x) =
    \begin{cases}
    \frac{1}{2}  exp((-x)/2) \ \ \ x > 0 \\
    0 \ \ \ otherwise
    \end{cases}
\end{align}

Plot CDF of $X$ : \\


\begin{tikzpicture}
\begin{axis}[
    axis lines = left,
    xlabel = $X$,
    ylabel = {$F_X(x)$},
]
\addplot[
domain = 1:10,
samples=100,
color=red,]
{(1-e^((-x)/2))};

%Here ends the first plot
\end{axis}
\end{tikzpicture}
\hskip 5pt
%Here begins the 2nd plot
\\ \\ \\ \\ \\ \\ \\ \\PDF of  $X$ - \\

\begin{tikzpicture}
\begin{axis}[
 xlabel=$X$,
 ylabel= {$f_X(x)$}]
\addplot[
domain= 1:10,
samples=100,
color=red,
]
{exp((-x)/2)/2};
\end{axis}
\end{tikzpicture}\\






\end{document}